%% SPECIFICA TECNICA

%controllare gli svantaggi

\section{Introduzione}
\subsection{Scopo del documento}

Questo documento ha come scopo quello di definire la progettazione ad alto livello
per il prodotto Monolith. Verrà presentata l’architettura generale secondo la quale
saranno organizzate le varie componenti software e i Design Pattern utilizzati nella
creazione dell'SDK, delle bolle predefinite e della demo. Verrà inoltre dettagliato il tracciamento tra le componenti software individuate ed i requisiti.


\subsection{Scopo del prodotto}

Lo scopo del prodotto è quello di permettere la creazione di bolle
interattive, che dovranno funzionare nell’ambiente Rocket.chat. Queste
bolle permetteranno di aumentare l'interattività tra gli utenti della
chat e aggiungeranno nuove funzionalità accessibili direttamente dalla conversazione 
senza il bisogno di ricorrere all'apertura di applicazioni diverse.
Il sistema offrirà agli sviluppatori un set di \glossario{API} per creare e
rilasciare nuove bolle e agli utenti finali la possibilità di
usufruire di un insieme di bolle predefinite.


\subsection{Glossario}

Al fine di evitare ogni ambiguità di linguaggio e massimizzare la
comprensione dei documenti, i termini che necessitano di essere
chiariti saranno scritti in corsivo e marcati con una |G| in pedice alla prima
occorrenza e saranno riportati nel Glossario.

\subsection{Riferimenti}

\subsubsection{Normativi}
\begin{itemize}
	\item \textbf{Norme di Progetto}: \\ NormediProgetto\_v1.1.0
	\item \textbf{Capitolato d'appalto C5}: \\ \url{http://www.math.unipd.it/~tullio/IS-1/2016/Progetto/C5.pdf}
	\item \textbf{Analisi dei Requisiti}: \\ AnalisideiRequisiti\_v1.0.0
	
\end{itemize}


\subsubsection{Informativi}
\begin{itemize}
	\item \textbf{Slide del corso di Ingegneria del Software}: \\  \url{http://www.math.unipd.it/~tullio/IS-1/2016/ }
\end{itemize}

\section{Tecnologie Utilizzate}

In questa sezione verranno descritte le tecnologie su cui si basa lo sviluppo del progetto. Per ognuna di esse, verranno indicati l’ambito di utilizzo della tecnologia, i vantaggi e gli svantaggi che ne derivano.Alcune delle tecnologie che saranno usate sono richieste come requisito dal capitolato scelto.

\subsection{Javascript 6th edition (ECMA SCRIPT 6)}

JavaScript è un linguaggio di scripting orientato agli oggetti e agli eventi. \'E comunemente utilizzato nella programmazione Web lato client per la creazione, in siti web e applicazioni web, di effetti dinamici interattivi tramite l'uso di funzioni di script invocate da eventi innescati in vari modi dall'utente sulla pagina web in uso. \\ Come richiesto dal capitolato, per la realizzazione di Monolith, deve essere utilizzato Javascript 6th edition (ECMA SCRIPT 6). \\

\textbf{Licenza}:  \\
Non esiste una sola implementazione perché ECMAScript (o ES) è un linguaggio di programmazione standardizzato e mantenuto da Ecma International nell'ECMA-262 ed ISO/IEC 16262.


\textbf{Vantaggi}: 
\begin{itemize}
	\item Gestione degli eventi asincroni tramite le promises
	\item Possibilità di dichiarare classi
	\item Supporto per le costanti(\emph{const})
	\item Possibilità di isolare la definizione di variabili ad un blocco (\emph{let})
	\item Possibilità di isolare lo scope di una funzione usando blocchi delimitati da parentesi graffe({}) come ambienti isolati (vs closure)
	\item Uso di sintassi più espressiva per scrivere le funzioni anonime (\emph{Arrow Functions})
	
\end{itemize}

\textbf{Svantaggi}: 
\begin{itemize}
	\item Il supporto di ES6 da parte dei browser è ancora incompleto
	\item L’assenza di tipizzazione potrebbe ostacolare la valutazione della correttezza del codice
\end{itemize}

\subsection{Meteor}

Meteor è un framework web JavaScript libero e open source  per lo sviluppo di applicazioni web e mobile. \'E una piattaforma basata su Node.js. Meteor utilizza, dunque, JavaScript sia lato client che lato server. 

\textbf{Licenza}: MIT \\
La licenza MIT è una delle licenze più permissive nel panorama open source. In modo più esplicito dichiara i diritti dati all'utente finale, incluso il diritto di utilizzare , copiare, modificare, incorporare, pubblicare, distribuire, sotto-licenziare, e/o vendere il software.


\textbf{Vantaggi}: 
\begin{itemize}
	\item Integrazione con diverse tecnologie utilizzate nello sviluppo web:
	\begin{itemize}
		\item React
		\item MongoDB
	\end{itemize}
	\item Isomorfismo: il codice javascript scritto funziona in modo trasparente sul client (browser), sul server (Node.js) o in entrambi i mondi
	\item Ecosistema e modularità: la comunità di Meteor è molto attiva e molte funzionalità client o server potrebbero già essere pacchettizzate dal package manager ufficiale. 
\end{itemize}

\textbf{Svantaggi}: %controllareeeeeee 
\begin{itemize}
	\item Inizialmente sconosciuto ai membri del gruppo.
	%\item Il modo in cui alcuni componenti core della tecnologia si interfacciano potrebbe limitare la libertà degli sviluppatori.
\end{itemize}

\subsection{Mongo DB}
MongoDB è un database NoSQL orientato ai documenti, basato sul formato JSON per la memorizzazione e la rappresentazione dei dati. \'E distribuito come software libero open source. \\

\textbf{Licenza}: GNU AGPL v3.0 \\
\'E una licenza pubblicata da Free Software Foundation. \'E simile alla capostipite GNU GPL, una licenza fortemente copyleft per software libero.

\textbf{Vantaggi}: 
\begin{itemize}
	\item \'E più flessibile di un database SQL e facilita la rappresentazione su un modello ad
	oggetti
	\item Supporta ricerche per campi, intervalli e regular expression. Le query possono restituire campi specifici del documento e anche includere funzioni definite dall'utente in JavaScript.
	\item Qualunque campo in MongoDB può essere indicizzato 
\end{itemize}


\textbf{Svantaggi}: 
\begin{itemize}
	\item Inizialmente sconosciuto ai membri del gruppo.	
\end{itemize}

\subsection{HTML5}
HTML5 è un linguaggio di markup per la strutturazione delle pagine web.

\textbf{Licenza}:  \\
Non esiste una sola implementazione perché HTML5 è un linguaggio di markup standardizzato e mantenuto da W3C.


\textbf{Vantaggi}: 
\begin{itemize}
	\item Codice più pulito e sintassi semplificata rispetto alle versioni precedenti
	\item Interattività senza l’ausilio di plugin esterni valida per diversi formati multimediali
	\item Semantica intuitiva grazie ai nuovi TAG di formattazione
	\item Introduzione della geolocalizzazione, dovuta ad una forte espansione di sistemi operativi mobili
	\item Sistema più efficiente alternativo ai normali cookie chiamato Web Storage 

\end{itemize}

\textbf{Svantaggi}: 
\begin{itemize}
	\item Non tutti i browser supportano HTML5
\end{itemize}

\subsection{SCSS}

SCSS è una sintassi per i fogli di stile introdotta da Sass 3 (Syntactically Awesome StyleSheets). \'E un'estensione del CSS .

\textbf{Licenza}: MIT \\
La licenza MIT è una delle licenze più permissive nel panorama open source. In modo più esplicito dichiara i diritti dati all'utente finale, incluso il diritto di utilizzare , copiare, modificare, incorporare, pubblicare, distribuire, sotto-licenziare, e/o vendere il software.

\textbf{Vantaggi}: 
\begin{itemize}
	\item Possibilità di utilizzare variabili
	\item Possibilità di creare funzioni 
	\item Possibilità di organizzare il foglio di stile in più file
	\item Compatibilità completa con la sintassi del CSS
\end{itemize}

\textbf{Svantaggi}: 
\begin{itemize}
	\item  Sintassi più complessa.
\end{itemize}

\subsection{React}

React è una libreria Javascript open source che permette di costruire interfacce utente. 

\textbf{Licenza}: BSD-3-Clause \\
Le licenze BSD sono una famiglia di licenze permissive, senza copyleft, per software.
Le tre clausole della licenza BSD-3-Clause sono:

\begin{itemize}
	\item Libertà di eseguire il programma per qualsiasi scopo
	\item Libertà di studiare il programma e modificarlo
	\item Libertà di ridistribuire copie del programma in modo da aiutare il prossimo
	
\end{itemize}

\textbf{Vantaggi}: 
\begin{itemize}
	\item Semplificazione della realizzazione di interfacce UI dinamiche che possono reagire ai cambiamenti di dati in maniera autonoma attraverso opportuni componenti
	\item Possibilità di utilizzare le viste per creare codice più facile da comprendere e su cui è più semplice effettuare il debugging.
	
\end{itemize}

\textbf{Svantaggi}: 
\begin{itemize}
	\item Implementa solo il livello view è quindi necessario utilizzare altre librerie per implementare altre parti dell'applicazione
	\item Curva di apprendimento ripida
	\item \'E una libreria relativamente nuova
\end{itemize}

\subsection{Node.js}

Node.js è una piattaforma event-driven per il motore JavaScript V8. Essa permette di realizzare applicazioni web utilizzando il linguaggio JavaScript, tipicamente client-side, per la scrittura anche della parte server-side.

\textbf{Licenza}: MIT \\
La licenza MIT è una delle licenze più permissive nel panorama open source. In modo più esplicito dichiara i diritti dati all'utente finale, incluso il diritto di utilizzare , copiare, modificare, incorporare, pubblicare, distribuire, sotto-licenziare, e/o vendere il software.

\textbf{Vantaggi}: 
\begin{itemize}
	\item Facile apprendimento
	\item Possibilità di realizzare applicazioni server-side senza dover imparare linguaggi di programmazione “tradizionali”
\end{itemize}

\textbf{Svantaggi}: 
\begin{itemize}
	\item Non supporta database relazionali
	
\end{itemize}

\subsection{Rochet.chat}

Rocket.chat è una Web chat server sviluppata in Javascript utilizzando il \glossario{Framework} Meteor.

\textbf{Licenza}: MIT \\
La licenza MIT è una delle licenze più permissive nel panorama open source. In modo più esplicito dichiara i diritti dati all'utente finale, incluso il diritto di utilizzare , copiare, modificare, incorporare, pubblicare, distribuire, sotto-licenziare, e/o vendere il software.

\textbf{Vantaggi}: 
\begin{itemize}

	\item Codice open source
	\item Possibilità di creare chat di gruppo
	\item Possibilità di inviare audio, video e file
	\item Possibilità di effettuare video chiamate
	\item Community molto attiva

	
\end{itemize}

\textbf{Svantaggi}: 
\begin{itemize}
	\item Parzialmente documentata
\end{itemize}

\subsection{Bootstrap}

Bootstrap è una raccolta di strumenti liberi per la creazione di siti e applicazioni per il Web. Essa contiene modelli di progettazione basati su HTML e CSS, sia per la tipografia, che per le varie componenti dell'interfaccia, come moduli, pulsanti e navigazione, così come alcune estensioni opzionali di JavaScript.

\textbf{Licenza}: MIT \\
La licenza MIT è una delle licenze più permissive nel panorama open source. In modo più esplicito dichiara i diritti dati all'utente finale, incluso il diritto di utilizzare , copiare, modificare, incorporare, pubblicare, distribuire, sotto-licenziare, e/o vendere il software.

\textbf{Vantaggi}: 
\begin{itemize}
	
	\item Piattaforma ben standardizzata 
	\item Non richiede l’appoggio né di un linguaggio di programmazione server side, né di un database
	\item Ottima documentazione
	\item Responsive Design	
	\item \'E supportato dai browser moderni
	
\end{itemize}

\textbf{Svantaggi}: 
\begin{itemize}
	\item I plugin di jQuery sono limitati
	\item Le modifiche dovute al continuo sviluppo non sono sempre facili da integrare
\end{itemize}


\section{Descrizione Architettura}
\subsection{Metodo e formalismo di specifica}

Nell’esposizione dell’architettura dell’applicazione si procederà con un approccio top-down, descrivendo l’architettura iniziando dal generale ed andando al particolare.
Si procederà quindi alla descrizione dei package, per poi descrivere
nel dettaglio le singole classi, specificando per ognuna il tipo, l’obiettivo, la funzione e
le relazioni in ingresso ed in uscita.
Successivamente si illustreranno degli esempi di uso dei Design Pattern nell’architettura del sistema, rimandando la spiegazione generale alla sezione dedicata.
L'architettura dell' SDK e della demo sono state progettate separatamente.  

 
\subsection{Architettura generale}


