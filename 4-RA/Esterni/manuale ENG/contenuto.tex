\section{Introduction}
\subsection{intro}
The following document explain how to install and utilize Monolith SDK.
It provide code snippet and example on how to create a custom bubble on Rocket.chat using the SDK.

\subsection{How to install}
To install Monolith SDK on Rocket.chat you need to follow the following steps:
\begin{enumerate}
    \item git clone https://github.com/RocketChat/Rocket.Chat.git
    \item cd Rocket.Chat
    \item meteor npm start \\ when finished close meteor
    \item meteor add templating blaze-html-templates react-meteor-data maxharris9:classnames react-template-helper
    \item meteor npm i react react-dom bluebird simpl-schema react-addons-pure-render-mixin money request request-promise  --save
    \item copy monolith-sdk folder inside the packages folder on Rocket.chat
    \item meteor add monolith-sdk
    \item meteor
\end{enumerate}

\begin{flushleft}
If you want to use an existing Rocket.chat application start from step 4.
\end{flushleft}

\section{Class Description}
This section explains how to use the Monolith library classes.

\subsection{SingleComponents}
\begin{flushleft}
Singlecomponent classes represent the components that can be rendered.

\paragraph{CheckButton}
CheckButton represent a HTML <checkbox> tag.
\begin{verbatim}
<CheckButton
    id="HTML id"
    classes="CSS classes"
    getCheck={this."function name"}
    value="checkbox value"
/>
\end{verbatim}
getCheck is a props that hold a function called when the checkbox onChange event is called and passes to the function a variable containing the state of the checkbox.
\paragraph{CheckBoxList}
CheckBoxList represent a group of CheckButton.
CheckBoxList needs to have an array passed like this:
\begin{verbatim}
let opt=[{id: 1, value: 'Hello World'},{id: 2, value: 'Installation'}];

<CheckBoxList
    classes="CSS classes"
    options={opt}
    getCheck=this."function name"
/>
\end{verbatim}
getCheck like the CheckButton getCheck.

    \paragraph{ComboBox}
ComboBox represent a HTML <select> tag.
\begin{verbatim}
<ComboBox
    id="HTML id"
    classes="CSS classes"
    options={["a","b","c"]}
    getSelection={this."function name"}
/>
\end{verbatim}
getSelection is a props that hold a function called when the select onChange event is called and passes to the function a variable containing the selected option.
    \paragraph{Image}
Image represent a HTML <img> tag.

\begin{verbatim}
<Image
    id="HTML id"
    classes="CSS classes"
    src="img source location"
    alt="image description"
    width="image width"
    height="image height"
/>
\end{verbatim}

    \paragraph{ImageButton}
ImageButton represent a button with an image.
\begin{verbatim}
<ImageButton
    id="HTML id"
    src="img source location"
    alt="image description"
    width="image width"
    height="image height"
    handleClick={this."function name"}
/>
\end{verbatim}
handleClick is a props that hold a function called when the ImageButton is clicked.

    \paragraph{LineEdit}
LineEdit represent a HTML text <input> tag.
\begin{verbatim}
<LineEdit
    id="HTML id"
    classes="CSS classes"
    updateState={this."function name"}
    value="default value"
 />
\end{verbatim}
updateState is a props that hold a function called when onChange event of the text input is called.

    \paragraph{LineEditComboBox}
LineEditComboBox represent a HTML text <input> and a HTML <select> tag.
\begin{verbatim}
<LineEditComboBox
    idle="LineEdit HTML id"
    idcb="ComboBox  HTML id"
    classesle="LineEdit CSS classes"
    classescb="ComboBox CSS classes"
    textUpdate={this."function name"}
    options={["a","b","c"]}
    comboUpdate={this."function name"}
/>
\end{verbatim}
textUpdate is a props that hold a function passed to the LineEdit.
comboUpdate is a props that hold a function passed to le ComboBox.

    \paragraph{PushButton}
PushButton represent a HTML <button>.
\begin{verbatim}
<PushButton
    id="HTML id"
    classes="CSS classes"
    handleClick={this."function name"}
    buttonName="button name"
/>
\end{verbatim}
handleClick is a props that hold a function called when the button is clicked.

    \paragraph{LineEditPushButton}
LineEditPushButton represent LineEdit and a PushButton.
\begin{verbatim}
<LineEditPushButton
    idle="LineEdit HTML id"
    idpb="PushButton HTML id"
    classesle="LineEdit CSS classes"
    classespb="PushButton CSS classes"
    getText={this."function name"}
    buttonName="button name"
/>
\end{verbatim}
getText is a props that hold a function called when the PushButton is clicked and passes to the function a variable containing the text of the LineEdit.

    \paragraph{RadioButtonGroup}
RadioButtonGroup represent a group of HTML radio button <input>.
\begin{verbatim}
<RadioButtonGroup
    classes="CSS classes"
    options={["a","b","c"]}
    getValue={this."function name"}
/>
\end{verbatim}
getValue is a props that hold a function called when a radio button is clicked and passes to the function a variable containing the selected radio information.

    \paragraph{TextAreaButton}
TextAreaButton represent a HTML <textarea> tag and a PushButton.
\begin{verbatim}
<TextAreaButton
    idta="textArea HTML id"
    classesta="textArea CSS classes"
    idpb="PushButton HTML id"
    classespb="PushButton CSS classes"
    getText={this."function name"}
    width="textarea width"
    height="textarea height"
    buttonName="button name"
/>
\end{verbatim}
getText is a props that hold a function called when the PushButton is clicked and passes to the function a variable containing the text of the textarea.

    \paragraph{TextAreaComboBox}
TextAreaComboBox represent a HTML <textarea> tag and a ComboBox.
\begin{verbatim}
<TextAreaComboBox
    idtx="textArea HTML id"
    classestx="textArea CSS classes"
    idcb="combobox HTML id"
    classescb="combobox CSS classes"
    width="textarea width"
    height="textarea height"
    textUpdate={this."function name"}
    options={["a","b","c"]}
    comboUpdate={this."function name"}
/>
\end{verbatim}
textUpdate is a props that hold a function called when onChange event of the textarea is called.\\
comboUpdate is a props that hold a function called when the select onChange event is called and passes to the function a variable containing the selected option.
\end{flushleft}

\subsection{Layout}

Layout are classes that represent are containers that place the elements conteined in a certain way.

\paragraph{VerticalLayout}
VerticalLayout represent a container that place the elements contained in a vertically.

\begin{verbatim}
<VerticalLayout hide={"visibility state(true or false)"}>
    <Children/>
    <Children/>
    .
    .
    .
</VerticalLayout>
\end{verbatim}

\paragraph{HorizontalLayout}
HorizontalLayout represent a container that place the elements contained in a horizontally.\\
The maximum number of element that can be displayed horizontally is 12.

\begin{verbatim}
<HorizontalLayout hide={"visibility state(true or false)"}>
    <Children/>
    <Children/>
    .
    .
    .
</HorizontalLayout>
\end{verbatim}
