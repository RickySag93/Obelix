\documentclass[10 pt,a4paper, openany]{article}
%titlepage
\usepackage[hidelinks]{hyperref}
\usepackage[italian]{babel}
\usepackage[T1]{fontenc}
\usepackage[utf8x]{inputenc}
\usepackage{amsfonts}
\usepackage{multicol}
\usepackage{graphicx}
\usepackage{amsmath}
\usepackage{framed}
\usepackage{extarrows}
\usepackage{cancel}
\usepackage{eurosym}
\usepackage{listingsutf8}
\usepackage{lastpage}
\usepackage{rotating} 
\usepackage{multirow}

\usepackage{caption}
\usepackage{makecell}
\usepackage{longtable}
\usepackage{array}
\date{}


\usepackage{makeidx}
\makeindex
\usepackage{fancyhdr}
\pagestyle{fancy}
\lhead{\includegraphics[width=.6cm]{../../../file_comuni/immagini/obelisk_sample_02.png}
  Obelix Group}
\chead{}
\rhead{\rightmark }% \leftmark}%da rimettere
\lfoot{}
\cfoot{}
\rfoot{\thepage / \pageref*{LastPage}}
%%%%%%%%%%%%%%%%%%%%%%%%%%%%%%%%%%%%%%
%\usepackage{lipsum}
\usepackage{../../../file_comuni/copertina}
\nomedoc{Verbale esterno 2017-03-01}
\versione{v1\_0\_0}
\datacreazione{2017/03/01}
\verifica{Riccardo Saggese}
\approvazione{Nicolò Rigato}
\redazione{Emanuele Crespan \eanche Federica Schifano}
\uso{interno}
\sommario{Verbale dell'incontro tra il gruppo \emph{Obelix} e il
  proponente \emph{RedBabel} in data 2017-03-01}


\begin{document}
%\paginatitolo
\section{Informazioni sulla riunione}

\begin{itemize}
\item[] Data: 2017-03-01
\item[] Luogo: Residenza membro gruppo Obelix e sede Red Babel - videoconferenza
\item[] Ora: 20:00
\item[] Durata: 45'
\item[] Partecipanti interni: Obelix
  \begin{itemize}
  \item[] Emanuele Crespan
  \item[] Federica Schifano
  \item[] Nicolò Rigato
  \item[] Riccardo Saggese
  \item[] Silvio Meneguzzo
  \item[] Tomas Mali
  \end{itemize}
\item[] Partecipanti esterni: Red Babel
  \begin{itemize}
  \item[] Alessandro Maccagnan
  \item[] Milo Ertola
  \end{itemize}
\end{itemize}

\section{Domande e risposte}

Di seguito vengono riportate in grassetto le domande effettuate dal gruppo Obelix
nel corso della videoconferenza e in corsivo le risposte date dal Proponente Red Babel.\\

\textbf{Quali documenti dovranno essere stilati in inglese?}\\
\textit{Il capitolato chiede due cose: la libreria SDK che serve a fare delle bolle e una
  demo che utilizza delle librerie per fare delle bolle complesse in un caso d'uso
  complesso, non banale. Tutto ciò che riguarda l'uso dell'SDK e della demo
  va scritto in inglese (ad es.il manuale dell'utente). Comunicazione essenziale e funzionale!}\\

\textbf{A proposito dell'SDK, vorremmo capire esattamente cosa ci si
  aspetti che venga fatto, dobbiamo intenderla come una libreria o come una sorta di template customizabile?}\\
\textit{SDK sta per Software Development Kit, cioè un insieme di cose che qualcun
  altro, cioè voi nel caso della demo dovete utilizzare per produrre delle bolle interattive
  e non banali. Un esempio di bolla interattiva e non banale è il sondaggio
  che prevede degli input, degli output, uno stato da mantenere e dei vari
  permessi. Potrebbe tornare utile a chi scrive il sondaggio avere già delle
  funzionalità pronte, tipo il layout, mantenere lo stato in qualche modo o
  quella dei permessi. In base a quello che voi decidete, questa SDK (in base a
  ciò che mette a disposizione) può essere un template o una libreria. Un altro
  dei gruppi gestirà principalmente la parte di layout e meno la parte di gestione
  dei dati e dei permessi, ma questo sta a voi.\\
  Tenete presente che il capitolato consta di due elementi, la parte tecnologica
  che è l'SDK, per produrre delle bolle con alcune bolle di esempio incluse
  e la demo che utilizza l'SDK per fare qualcosa di non banale.\\
  Dunque siete liberi, ma con certi vincoli tecnologici e stilistici. Avete vincoli
  su come scrivere le cose, non tanto su cosa fare (ad es.niente call back).}\\

\textbf{Vi interessa di più la demo o l'SDK?}\\
\textit{A noi interessano entrambe. Con l'SDK c'è un lavoro più architetturale,
  tecnico, mentre nella demo c'è un altro tipo di lavoro che è un po' più di
  fantasia. L'idea è che con la demo vi mettiate nei panni di persone diverse
  da voi che hanno un problema e che capiate come risolvere questo problema con
  quello che avete fatto. L'idea è: ho inventato qualcosa e ne creo un utilizzo per
  qualcuno. Questo è il tipo di lavoro che vogliamo che facciate, bisogna andare
  oltre l'aspetto puramente tecnico (capacità che sviluppate con i progetti
  dell'università) e usare quello che avete imparato per fare un prodotto.}\\

\textbf{Potete farci un esempio di prodotto che potrebbe essere realizzato?}\\
\textit{Vi do un esempio negativo, un esempio di ciò che NON dovete fare: avete creato
  l'algoritmo più veloce del mondo per ordinare i numeri, ma l'algoritmo è
  inutilizzabile, non è presentato, non c'è scritta nessuna complessità e la gente
  non lo usa. Questo è quello che non dovete fare, vi chiedamo di fare invece
  un prodotto che la gente può utilizzare.\\
  Vari esempi positivi di bolle si evincono dal capitolato e dal link cola.io
  che abbiamo allegato al capitolato, dove sono presentati dei casi di bolle che
  possono essere utilizzate nella chat, su whatsapp piuttosto che su facebook.}\\

\textbf{Un esempio che ci viene in mente è il bot di telegram}\\
\textit{Si, è un buon esempio.
  Per portare avanti il progetto è comunque necessario possedere le softskill
  (comunicazione efficace ed entusiasmo), oltre alle hard skill (quelle di programmazione).}\\

\textbf{Preferite che sviluppiamo alcune delle bolle tra quelle da voi proposte o delle bolle che abbiamo pensato noi? Ci sono delle bolle "necessarie" da realizzare?}\\
\textit{A noi non interessa molto delle bolle predefinite ma ci interessa che quelle
  che voi pensate abbiano un caso d'uso forte. Le bolle che fate sono legate al
  caso d'uso, sta a voi decidere a che livello di astrazione spingere l'SDK. Vi
  faccio un esempio: potete fare un SDK che vi permette di fare tutte le form,
  oppure potete fare un SDK che vi permette di mettere diverse bolle insieme,
  comporre le bolle. Sta a voi decidere il livello di astrazione: mettere nell'SDK
  delle bolle predefinite o nell'SDK fare solo quell'insieme di regole che
  vi permettono di creare le bolle che volete.\\
  Quello che vogliamo è che in ogni singola bolla deve trasparire il vostro spirito!}\\

\textbf{Per quanto riguarda le specifiche tecnologiche alcune sono raccomandate altre sono obbligatorie, giusto?}\\
\textit{Quelle sicuramente obbligatorie sono: non usare callbacks ma promises, usare
  Javascript 6. Obbligatori sono i punti: 1, 2, 5 e 6 della sezione 4.1 pag 7
  del capitolato. L'SCSS è assolutamente da usare, probabilmente Meteor lo supporta già.
  Consigliamo di seguire una serie di regole che si chiama GitFlow per usare
  git, se vi attenete a quello non avrete problemi di gestione del codice.}\\

\textbf{Come ci organizzeremo per le riunioni successive?}\\
\textit{Potremmo fare altre call su Skype, ma possiamo continuare a sentirci tramite
    messaggi su Slack. Probabilmente riusciremo anche ad organizzarci per un
    incontro fisico a fine Aprile.}
\end{document}
