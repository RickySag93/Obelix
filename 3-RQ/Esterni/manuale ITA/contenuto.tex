%% MANUALE DELL'APPLICAZIONE
\section{Introduzione}
\subsection{Cos'è Monolith SDK?}
Monolith SDK è un pacchetto \glossario{Meteor} che consente la creazione di bolle interattive in ambiente \glossario{Rocket.chat}.\\
Lo Sviluppatore avrà la possibilità di usare i componenti della SDK per costruire le proprie bolle da integrare a quelle già esistenti sulla piattaforma di \glossario{Rocket.chat}. 
\subsection{Scopo del documento}
Questo documento rappresenta il manuale utente per l'applicazione Monolith SDK nel quale vengono descritte dettagliatamente tutte le caratteristiche dell'applicativo utilizzabili dall'utente.
Il manuale sarà diviso in sezioni per essere maggiormente comprensibile e spiegherà le varie componenti che l'utente potrà utilizzare per costruire la propria bolla.
\subsection{Glossario}
In modo tale da evitare ambiguità di linguaggio e massimizzare la comprensione, i termini tecnici, di dominio, gli acronimi e le parole che necessitano di essere chiarite, sono riportate nel documento \gloss .\\
Ogni occorrenza di vocaboli presenti nel Glossario è marcata da una \glossario{} maiuscola in pedice.

\section{Utilizzo}

\subsection{Layout}
\subsubsection{VerticalLayout}
Layout verticale che conterrà i componenti della bolla e li posizionerà ognuno sotto al proprio precedente. La props \textit{hide} permette di nascondere la vista del layout (tramite valori \textit{true/false}).
\begin{center}
\underline{\textit{Utilizzo}}
\begin{lstlisting}
<VerticalLayout hide={}>
<Children/>
<Children/>
.
.
.
</VerticalLayout>
\end{lstlisting}
\end{center}

\subsubsection{HorizontalalLayout}
Layout orizzontale che conterrà i componenti della bolla e li posizionerà ognuno di fianco al proprio precedente. La props \textit{hide} permette di nascondere la vista del layout (tramite valori \textit{true/false}).
\begin{center}
\underline{\textit{Utilizzo}}
\begin{lstlisting}
<HorizontalLayout hide={}>
<Children/>
<Children/>
.
.
.
</HorizontalLayout>
\end{lstlisting}
\end{center}

\subsection{Components}
\subsubsection{CheckBoxList}
Una lista di Checkbox, i dati del CheckButton cliccato vengono ritornati tramite la funzione getCheck.
\begin{center}
\underline{\textit{Utilizzo}}
\begin{lstlisting} 

<CheckBoxList 
classes= // CSS classes 
options={opt} 
getCheck={this.fun}
/>
\end{lstlisting}
\underline{\textit{Esempio opzioni}}
\begin{lstlisting}
var opt=[{id: 1, value: 'Hello World'},{id: 2, value: 'value'}];
\end{lstlisting}
\underline{\textit{Funzione di ritorno}}
\begin{lstlisting}
function fun(m) {...}

m={id:'', value:'', check:''};
\end{lstlisting}
\end{center}

\subsubsection{CheckButton}

Componente Checkbox, i suoi dati vengono ritornati (ad ogni click) tramite la funzione getCheck.
\begin{center}	
\underline{\textit{Utilizzo}}
\begin{lstlisting}
<CheckButton
id= ""
classes= // CSS classes
getCheck={this.fun} 
value="" // Valore CheckButton
/>
\end{lstlisting}
\underline{\textit{Funzione di ritorno}}
\begin{lstlisting}
function fun(m) {...}

m={id:'', value:'', check:''};
\end{lstlisting}
\end{center}

\subsubsection{ComboBox}
Semplice elemento ComboBox. La funzione di ritorno riceve l'elemento selezionato.
\begin{center}
\underline{\textit{Utilizzo}}
\begin{lstlisting}	
<ComboBox
id= ""
classes= // CSS classes
options={["a","b","c"]} //array of options
getSelection={this.fun}
/>
\end{lstlisting}
\end{center}

\subsubsection{Image}
Semplice elemento Immagine senza funzione di ritorno.
\begin{center}
\underline{\textit{Utilizzo}}
\begin{lstlisting}
<Image
id= ""
classes= // CSS classes
src= // like HTML "src" attribute
alt=  // like HTML "alt" attribute
width=  // like HTML "width" attribute
height=  // like HTML "height" attribute
/>
\end{lstlisting}
\end{center}

\subsubsection{ImageButton}
Elemento Immagine cliccabile come pulsante. La funzione di ritorno viene avviata quando l'immagine viene cliccata.
\begin{center}
\underline{\textit{Utilizzo}}
\begin{lstlisting}
<ImageButton
id= ""
src= // like HTML "src" attribute
alt=  // like HTML "alt" attribute
width=  // like HTML "width" attribute
height=  // like HTML "height" attribute
handleClick={this.fun}
/>
\end{lstlisting}
\end{center}


\subsubsection{LineEdit}
Casella di testo editabile. La funzione "fun" salva il testo digitato e viene chiamata ad ogni modifica.
\begin{center}
\underline{\textit{Utilizzo}}
\begin{lstlisting}
<LineEdit 
id= ""
classes= // CSS classes
updateState={this.fun}
value="default value"
/>
\end{lstlisting}
\end{center}

\subsubsection{LineEditComboBox}
Elemento formato da una casella di testo editabile e una ComboBox.\\
La funzione "fun1" riceve il testo dalla LineEdit mentre "fun2" l'opzione selezionata dal ComboBox.
\begin{center}
\underline{\textit{Utilizzo}}
\begin{lstlisting}	
<LineEditComboBox
idle= // lineEdit id
idcb= // comboBox id
classesle= // lineEdit CSS classes
classescb= // comboBox CSS classes
textUpdate={this.fun1}
options={["a","b","c"]} //array of option
comboUpdate={this.fun2}
/>
\end{lstlisting}
\end{center}

\subsubsection{LineEditPushButton}
Elemento formato da una casella di testo editabile e un pulsante cliccabile.\\
La funzione "fun" riceve il testo dalla LineEdit quando il PushButton viene cliccato.
\begin{center}
\underline{\textit{Utilizzo}}
\begin{lstlisting}
<LineEditPushButton
idle= // lineEdit id
idpb= // pushButton id
classesle= // lineEdit CSS classes
classespb= // pushButton CSS classes
getText={this.fun}
buttonName="button name"
/>	
\end{lstlisting}
\end{center}

\subsubsection{PushButton}
Pulsante cliccabile. La funzione "fun" viene eseguita al click sul pulsante.
\begin{center}
\underline{\textit{Utilizzo}}
\begin{lstlisting}
<PushButton
id= ""
classes= // CSS classes
handleClick={this.fun}
buttonName="button name"
/>	
\end{lstlisting}
\end{center}

\subsubsection{RadioButtonGroup}
Elemento formato da un gruppo di RadioButton. La funzione "fun" restituisce l'opzione selezionata ad ogni cambiamento.
\begin{center}
\underline{\textit{Utilizzo}}
\begin{lstlisting}
<RadioButtonGroup
classes= // CSS classes
options={["a","b","c"]} //array of options
getValue={this.fun}
/>
\end{lstlisting}
\end{center}

\subsubsection{TextAreaButton}
Elemento area di testo editabile ed un PushButton.\\
La funzione "fun" riceve il testo dalla TextArea quando il PushButton viene cliccato.
\begin{center}
\underline{\textit{Utilizzo}}
\begin{lstlisting}
<TextAreaButton
idta= // textArea id
classesta= // textArea CSS classes
idpb= // PushButton id
classespb= // CSS classes
getText={this.fun}
width= // textarea width
height= // textarea height
buttonName= // button name
/>
\end{lstlisting}
\end{center}

\subsubsection{TextAreaComboBox}
Elemento area di testo editabile ed un ComboBox.\\
La funzione "fun1" riceve il testo dalla TextArea mentre "fun2" l'opzione selezionata dal ComboBox.
\begin{center}
\underline{\textit{Utilizzo}}
\begin{lstlisting}
<TextAreaComboBox
idtx= //textArea id 
classestx= // textArea CSS classes
idcb= //combobox id 
classescb= // combobox CSS classes
width= // textarea width
height= // textarea height
textUpdate={this.fun1}
options={["a","b","c"]} // array of options
comboUpdate={this.fun2}
/>	
\end{lstlisting}
\end{center}