\subsection{Requisiti funzionali}

\begin{center}
\begin{longtable}{|
*{1}{>{\centering\arraybackslash}p{2.5cm}|}
*{1}{>{\centering\arraybackslash}p{2cm}|}
*{1}{>{\centering\arraybackslash}p{5cm}|}
*{1}{>{\centering\arraybackslash}p{2.5cm}|}}
\hline \textbf{Requisito} & \textbf{Tipologia} & \textbf{Descrizione} & \textbf{Fonti}\\
\hline \endhead
\hline \endfoot

RObFu10 & \makecell{Obbligatorio \\ Funzionale} & L'utente deve poter accedere alle bolle dall'interfaccia di Rocket.Chat. & \makecell{UC3}\\
\hline

RObFu10.1 & \makecell{Obbligatorio \\ Funzionale} & All'interfaccia standard di Rocket.Chat devono essere aggiunti due pulsanti nella tabbar laterale che permettano di accedere alle SideAreas. & \makecell{UC3.1\\UC3.4}\\
\hline

RObFu10.2 & \makecell{Obbligatorio \\ Funzionale} & L'utente deve poter utilizzare la SideArea1 che permette di creare, inviare, visualizzare e modificare le bolle inviate. & \makecell{UC3.2}\\
\hline

RObFu10.2.1 & \makecell{Obbligatorio \\ Funzionale} & L'utente deve poter visualizzare i tipi di bolla disponibili. & \makecell{UC3.2.1}\\
\hline

RObFu10.2.2 & \makecell{Obbligatorio \\ Funzionale} & L'utente deve poter selezionare il tipo di bolla da inviare. & \makecell{UC3.2.2}\\
\hline

RObFu10.2.3 & \makecell{Obbligatorio \\ Funzionale} & L'utente deve poter configurare la bolla tramite l'apposito menu. & \makecell{UC3.2.3}\\
\hline

RObFu10.2.4 & \makecell{Obbligatorio \\ Funzionale} & L'utente deve poter inviare la bolla che ha configurato. & \makecell{UC3.2.4}\\
\hline

RObFu10.2.5 & \makecell{Obbligatorio \\ Funzionale} & L'utente deve poter visualizzare lo storico delle bolle che ha inviato. & \makecell{UC3.2.5}\\
\hline

RObFu10.2.5.1 & \makecell{Obbligatorio \\ Funzionale} & L'utente deve visualizzare un messaggio appropriato nel caso non ci siano bolle nello storico in uscita. & \makecell{UC3.2.7}\\
\hline

RObFu10.2.7 & \makecell{Obbligatorio \\ Funzionale} & L'utente non deve poter inviare la bolla nel caso in cui la configurazione non sia andata a buon fine. & \makecell{Interno\\UC3.2.8}\\
\hline

ROpFu10.2.8 & \makecell{Opzionale \\ Funzionale} & L’utente deve poter visualizzare un messaggio di errore nel caso in cui la configurazione della bolla non fosse andata a buon fine. & \makecell{UC3.2.8}\\
\hline

RObFu10.3 & \makecell{Obbligatorio \\ Funzionale} & L'utente deve poter utilizzare la SideArea2 che permette di visualizzare e interagire con le bolle ricevute. & \makecell{UC3.3}\\
\hline

RObFu10.3.1 & \makecell{Obbligatorio \\ Funzionale} & L'utente deve poter visualizzare lo storico delle bolle ricevute. & \makecell{UC3.3.1}\\
\hline

RObFu10.3.1.1 & \makecell{Obbligatorio \\ Funzionale} & L'utente deve visualizzare un messaggio appropriato nel caso in cui non ci siano bolle nello storico in ingresso. & \makecell{UC3.3.3}\\
\hline

RObFu11 & \makecell{Obbligatorio \\ Funzionale} & Lo sviluppatore deve avere a disposizione gli strumenti per creare una bolla funzionante. & \makecell{UC0}\\
\hline

RObFu11.1 & \makecell{Obbligatorio \\ Funzionale} & Lo sviluppatore deve poter creare le proprie bolle a partire dalla bolla vuota. & \makecell{UC1}\\
\hline

RObFu11.2 & \makecell{Obbligatorio \\ Funzionale} & Lo sviluppatore deve poter accedere alle funzionalità offerte dall'API. & \makecell{UC2}\\
\hline

RObFu11.2.1 & \makecell{Obbligatorio \\ Funzionale} & Lo sviluppatore deve poter utilizzare le funzionalità offerte dal sistema per descrivere l'aspetto visuale delle bolle. & \makecell{UC2.1}\\
\hline

RObFu11.2.1.1 & \makecell{Obbligatorio \\ Funzionale} & Lo sviluppatore deve poter inserire un componente grafico. & \makecell{UC2.1.1}\\
\hline

RObFu11.2.1.1.2 & \makecell{Obbligatorio \\ Funzionale} & Lo sviluppatore deve poter inserire un componente immagine. & \makecell{Interno\\UC2.1.1}\\
\hline

RObFu11.2.1.1.3 & \makecell{Obbligatorio \\ Funzionale} & Lo sviluppatore deve poter inserire un componente campo di inserimento testo. & \makecell{Interno\\UC2.1.1}\\
\hline

RObFu11.2.1.1.4 & \makecell{Obbligatorio \\ Funzionale} & Lo sviluppatore deve poter inserire un componente pulsante. & \makecell{UC2.1.1\\interno}\\
\hline

RObFu11.2.1.1.5 & \makecell{Obbligatorio \\ Funzionale} & Lo sviluppatore deve poter inserire un componente checkbox. & \makecell{Interno\\UC2.1.1}\\
\hline

RObFu11.2.1.1.6 & \makecell{Obbligatorio \\ Funzionale} & Lo sviluppatore deve poter inserire un componente radiobutton. & \makecell{Interno\\UC2.1.1}\\
\hline

RObFu11.2.1.1.7 & \makecell{Obbligatorio \\ Funzionale} & Lo sviluppatore deve poter inserire un combobox. & \makecell{Interno\\UC2.1.1}\\
\hline

RObFu11.2.1.2 & \makecell{Obbligatorio \\ Funzionale} & Lo sviluppatore deve poter inserire un contenitore. & \makecell{UC2.1.5}\\
\hline

RObFu11.2.1.2.1 & \makecell{Obbligatorio \\ Funzionale} & Lo sviluppatore deve poter inserire un contenitore che posiziona gli elementi figli uno accanto all'altro dando a ciascuno uguale spazio. Il massimo numero di elementi accostabili è 12 a causa dei limiti del grid system di Bootstap. & \makecell{Interno\\UC2.1.5}\\
\hline

RObFu11.2.1.2.2 & \makecell{Obbligatorio \\ Funzionale} & Lo sviluppatore deve poter inserire un contenitore che posiziona gli elementi figli uno sotto l'altro. & \makecell{Interno\\UC2.1.5}\\
\hline

RObFu11.2.1.2.3 & \makecell{Obbligatorio \\ Funzionale} & Lo sviluppatore deve poter inserire un contenitore il cui contenuto possa essere visualizzato o meno in base all'esecuzione di un comando. & \makecell{Interno\\UC2.1.5}\\
\hline

RObFu11.2.1.3 & \makecell{Obbligatorio \\ Funzionale} & Lo sviluppatore deve poter modificare le proprietà di un componente grafico. & \makecell{UC2.1.2}\\
\hline

RObFu11.2.1.3.1 & \makecell{Obbligatorio \\ Funzionale} & Lo sviluppatore deve poter impostare le opzioni di scelta di un radiobutton. & \makecell{UC2.1.2.6}\\
\hline

RObFu11.2.1.3.3 & \makecell{Obbligatorio \\ Funzionale} & Lo sviluppatore deve poter impostare il percorso dell'immagine da visualizzare in un componente immagine. & \makecell{UC2.1.2.2}\\
\hline

RObFu11.2.1.3.4 & \makecell{Obbligatorio \\ Funzionale} & Lo sviluppatore deve poter modificare il testo mostrato nel pulsante. & \makecell{UC2.1.2.3}\\
\hline

RObFu11.2.1.3.5 & \makecell{Obbligatorio \\ Funzionale} & Lo sviluppatore deve poter impostare l'azione da associare al pulsante. & \makecell{UC2.1.2.4}\\
\hline

RObFu11.2.1.3.6 & \makecell{Obbligatorio \\ Funzionale} & Lo sviluppatore deve poter impostare il testo della checkbox. & \makecell{UC2.1.2.5}\\
\hline

RObFu11.2.1.3.7 & \makecell{Obbligatorio \\ Funzionale} & Lo sviluppatore deve poter specificare per ciascun componente fornito dall'SDK la classe HTML. & \makecell{UC2.1.2.7}\\
\hline

RObFu11.2.1.3.8 & \makecell{Obbligatorio \\ Funzionale} & Lo sviluppatore deve poter specificare le opzioni di scelta di una combobox. & \makecell{UC2.1.2.8}\\
\hline

RObFu11.2.1.6 & \makecell{Obbligatorio \\ Funzionale} & Lo sviluppatore deve poter inserire il menu di configurazione della bolla. & \makecell{UC2.1.4}\\
\hline

RObFu11.2.1.7 & \makecell{Obbligatorio \\ Funzionale} & Lo sviluppatore deve poter selezionare un componente grafico o un contenitore. & \makecell{UC2.1.3}\\
\hline

RObFu11.2.2 & \makecell{Obbligatorio \\ Funzionale} & Il sistema deve distinguere il mittente di una bolla da tutti gli altri utenti. & \makecell{UC2.2}\\
\hline

RObFu11.2.3 & \makecell{Obbligatorio \\ Funzionale} & Lo sviluppatore deve poter gestire la persistenza dei dati della bolla. & \makecell{UC2.3}\\
\hline

RObFu11.2.3.1 & \makecell{Obbligatorio \\ Funzionale} & Lo sviluppatore deve poter impostare la memorizzazione di dati nella bolla. & \makecell{Interno\\UC2.3\\UC2.3.3}\\
\hline

RObFu11.2.3.2 & \makecell{Obbligatorio \\ Funzionale} & Lo sviluppatore deve poter impostare la modifica dei dati memorizzati nella bolla. & \makecell{Interno\\UC2.3\\UC2.3.3}\\
\hline

RObFu11.2.3.3 & \makecell{Obbligatorio \\ Funzionale} & Lo sviluppatore deve poter impostare l'eliminazione dei dati memorizzati nella bolla. & \makecell{Interno\\UC2.3\\UC2.3.3}\\
\hline

RObFu11.2.3.4 & \makecell{Obbligatorio \\ Funzionale} & Lo sviluppatore deve poter specificare i parametri di accettazione dell'input. & \makecell{UC2.3.1}\\
\hline

ROpFu11.2.3.5 & \makecell{Opzionale \\ Funzionale} & Lo sviluppatore deve poter specificare come gestire l'input non accettabile. & \makecell{UC2.3.2}\\
\hline

ROpFu11.2.3.5.1 & \makecell{Opzionale \\ Funzionale} & Lo sviluppatore deve poter specificare come il sistema deve gestire gli errori sull'input. & \makecell{UC2.3.2.1}\\
\hline

ROpFu11.2.3.5.2 & \makecell{Opzionale \\ Funzionale} & Lo sviluppatore deve poter specificare un messaggio d'errore per l'utente in caso di input errato. & \makecell{UC2.3.2.2}\\
\hline

RObFu20 & \makecell{Obbligatorio \\ Funzionale} & I dati relativi alle singole istanze di bolla vengono memorizzati separatamente senza che lo sviluppatore debba operare la distinzione. & \makecell{Interno}\\
\hline

RObFu21 & \makecell{Obbligatorio \\ Funzionale} & Ogni client riceve dal data-system di Meteor solo i dati che lo riguardano, ovvero quelli di tutte le bolle presenti in rooms in cui sia presente anche l'utente. & \makecell{Interno}\\
\hline

RObFu22 & \makecell{Obbligatorio \\ Funzionale} & Deve essere possibile per lo sviluppatore effettuare dei controlli sull'input dell'utente. & \makecell{Interno}\\
\hline

RObFu22.1 & \makecell{Obbligatorio \\ Funzionale} & Lo sviluppatore deve essere in grado di effettuare controlli specifici per ciascuna situazione di input da  parte dell'utente. & \makecell{Interno}\\
\hline

RObFu23 & \makecell{Obbligatorio \\ Funzionale} & Vengono forniti allo sviluppatore una serie di componenti grafici di base. & \makecell{Interno}\\
\hline

RDeFu24 & \makecell{Desiderabile \\ Funzionale} & L'utente deve riceve notifiche per eventi di rilievo. & \makecell{Interno}\\
\hline

RDeFu24.1 & \makecell{Desiderabile \\ Funzionale} & L'utente deve ricevere una notifica per la ricezione di una bolla. & \makecell{Interno}\\
\hline

RDeFu24.2 & \makecell{Desiderabile \\ Funzionale} & Lo sviluppatore deve poter impostare notifiche per eventi di rilievo per un nuovo tipo di bolla. & \makecell{Interno}\\
\hline

RObFu01-cv & \makecell{Obbligatorio \\ Funzionale} & L'utente può convertire importi da una valuta all'altra. & \makecell{UC0-cv}\\
\hline

RObFu01.1-cv & \makecell{Obbligatorio \\ Funzionale} & L'utente può scegliere le valute tra cui effettuare la conversione. & \makecell{UC1-cv}\\
\hline

RObFu01.2-cv & \makecell{Obbligatorio \\ Funzionale} & L'utente può inserire l'importo da convertire. & \makecell{UC2-cv}\\
\hline

RObFu01.3-cv & \makecell{Obbligatorio \\ Funzionale} & La conversione viene effettuata tramite le librerie money.js e request.js. & \makecell{Interno}\\
\hline

RObFu01.3.1-cv & \makecell{Obbligatorio \\ Funzionale} & I tassi di conversione vengono scaricati da openexchangerates.org tramite request.js. & \makecell{Interno}\\
\hline

RObFu01.3.2-cv & \makecell{Obbligatorio \\ Funzionale} & La conversione viene effettuata tramite money.js. & \makecell{Interno}\\
\hline

RObFu01.4-cv & \makecell{Obbligatorio \\ Funzionale} & Il mittente e il ricevente devono poter visualizzare gli importi convertiti. & \makecell{UC3-cv}\\
\hline

RObFu01.5-cv & \makecell{Obbligatorio \\ Funzionale} & L'utente deve poter visualizzare il menu di configurazione per la bolla convertitore di valuta. & \makecell{UC4-cv}\\
\hline

RObFu01.6-cv & \makecell{Obbligatorio \\ Funzionale} & L'utente deve poter inviare la bolla convertitore di valuta. & \makecell{UC5-cv}\\
\hline

RObFu01.7-cv & \makecell{Obbligatorio \\ Funzionale} & L'utente non deve poter inviare la bolla nel caso in cui non siano state selezionate le valute in ingresso e in uscita o l'importo da convertire. & \makecell{UC5-cv}\\
\hline

ROpFu02-cv & \makecell{Opzionale \\ Funzionale} & L'utente può convertire importi da valori di pacchetti azionari. & \makecell{Interno}\\
\hline

RObFu01-dd & \makecell{Obbligatorio \\ Funzionale} & La bolla estrae un numero casualmente dal range impostato. & \makecell{UC0-dd}\\
\hline

RObFu01.1-dd & \makecell{Obbligatorio \\ Funzionale} & L'utente deve poter selezionare un numero intero positivo tra quelli proposti. & \makecell{UC1-dd}\\
\hline

RObFu01.2-dd & \makecell{Obbligatorio \\ Funzionale} & Il mittente e il ricevente devono poter visualizzare il numero casuale generato. & \makecell{UC2-dd}\\
\hline

ROpFu01.2.1-dd & \makecell{Opzionale \\ Funzionale} & Il mittente e il ricevente devono poter visualizzare il numero casuale sotto forma di immagine (per esempio le facce di uno o più dadi). & \makecell{Interno}\\
\hline

RObFu01.3-dd & \makecell{Obbligatorio \\ Funzionale} & Il sistema genera lato server un numero intero casuale nel range tra 0 e il numero inserito. L'estrazione avviene utilizzando la libreria Math inclusa in Javascript. & \makecell{Interno}\\
\hline

RObFu01.4-dd & \makecell{Obbligatorio \\ Funzionale} & L'utente deve poter visualizzare il menu di configurazione della bolla estrazione di numero casuale. & \makecell{UC3-dd}\\
\hline

RObFu01.5-dd & \makecell{Obbligatorio \\ Funzionale} & L'utente deve poter inviare la bolla estrazione di numero casuale. & \makecell{UC4-dd}\\
\hline

RObFu01.6-dd & \makecell{Obbligatorio \\ Funzionale} & L'utente non deve poter inviare la bolla senza aver impostato il range. & \makecell{UC4-dd}\\
\hline

RObFu01-ls & \makecell{Obbligatorio \\ Funzionale} & Il mittente deve poter definire una lista da inviare. & \makecell{UC1-ls}\\
\hline

RObFu01.1-ls & \makecell{Obbligatorio \\ Funzionale} & Il mittente deve poter inserire manualmente un nuovo elemento. & \makecell{UC1.1-ls}\\
\hline

RObFu01.2-ls & \makecell{Obbligatorio \\ Funzionale} & Il mittente deve poter inserire un elemento prelevandolo da una delle liste predefinite. & \makecell{UC1.2-ls}\\
\hline

RObFu01.3-ls & \makecell{Obbligatorio \\ Funzionale} & L'utente deve poter inserire un titolo per la lista da inviare. & \makecell{UC1.3-ls}\\
\hline

RObFu01.4-ls & \makecell{Obbligatorio \\ Funzionale} & L'utente deve poter rimuove un elemento dalla lista. & \makecell{UC1.4-ls}\\
\hline

RObFu01.5-ls & \makecell{Obbligatorio \\ Funzionale} & L'utente deve poter modificare un elemento della lista. & \makecell{UC1.5-ls}\\
\hline

RObFu01.6-ls & \makecell{Obbligatorio \\ Funzionale} & L'utente deve poter visualizzare l'elenco delle checklists. & \makecell{UC1.6-ls}\\
\hline

RObFu01.7-ls & \makecell{Obbligatorio \\ Funzionale} & L'utente deve poter visualizzare la singola checklist. & \makecell{UC1.7-ls}\\
\hline

RObFu02-ls & \makecell{Obbligatorio \\ Funzionale} & Il mittente deve poter definire una lista predefinita. & \makecell{UC2-ls}\\
\hline

RObFu02.1-ls & \makecell{Obbligatorio \\ Funzionale} & L'utente deve poter accedere al menu di gestione delle checklists. & \makecell{UC2.1-ls}\\
\hline

RObFu02.2-ls & \makecell{Obbligatorio \\ Funzionale} & L'utente deve poter visualizzare la singola checklist. & \makecell{UC2.2-ls}\\
\hline

RObFu02.3-ls & \makecell{Obbligatorio \\ Funzionale} & L'utente deve poter accedere al menu di creazione di una nuova checklist. & \makecell{UC2.3-ls}\\
\hline

RObFu02.4-ls & \makecell{Obbligatorio \\ Funzionale} & L'utente non deve poter creare una checklist vuota. & \makecell{UC2.4-ls}\\
\hline

RDeFu02.4.1-ls & \makecell{Desiderabile \\ Funzionale} & L'utente deve visualizzare un messaggio che lo informa del non avvenuto salvataggio di una checklist vuota. & \makecell{UC2.4-ls}\\
\hline

RObFu02.5-ls & \makecell{Obbligatorio \\ Funzionale} & L'utente deve poter inserire un titolo alla checklist. & \makecell{UC2.5-ls}\\
\hline

RObFu02.6-ls & \makecell{Obbligatorio \\ Funzionale} & L'utente deve poter aggiungere un elemento alla checklist. & \makecell{UC2.6-ls}\\
\hline

RObFu02.7-ls & \makecell{Obbligatorio \\ Funzionale} & L'utente deve poter eliminare un elemento dalla checklist. & \makecell{UC2.7-ls}\\
\hline

RObFu02.8-ls & \makecell{Obbligatorio \\ Funzionale} & L'utente deve poter modificare un elemento della checklist. & \makecell{UC2.8-ls}\\
\hline

RObFu02.9-ls & \makecell{Obbligatorio \\ Funzionale} & L'utente deve poter eliminare una checklist. & \makecell{UC2.9-ls}\\
\hline

RObFu03-ls & \makecell{Obbligatorio \\ Funzionale} & Il mittente e il ricevente devono poter spuntare una voce dalla lista inviatagli. & \makecell{UC3-ls}\\
\hline

RObFu04-ls & \makecell{Obbligatorio \\ Funzionale} & Il mittente e il ricevente devono poter visualizzare le spunte effettuate. & \makecell{UC4-ls}\\
\hline

RObFu05-ls & \makecell{Obbligatorio \\ Funzionale} & L'utente deve poter visualizzare il menu di configurazione della bolla lista. & \makecell{UC5-ls}\\
\hline

RObFu06-ls & \makecell{Obbligatorio \\ Funzionale} & L'utente deve poter inviare la bolla lista. & \makecell{UC6-ls}\\
\hline

RObFu07-ls & \makecell{Obbligatorio \\ Funzionale} & L'utente non deve poter inviare una lista bolla senza averla definita con almeno un elemento. & \makecell{UC6-ls}\\
\hline

ROpFu08-ls & \makecell{Opzionale \\ Funzionale} & L'utente riceve una notifica al completamento di una lista. & \makecell{Interno}\\
\hline

ROpFu01-mt & \makecell{Opzionale \\ Funzionale} & La bolla deve restituire le previsioni meteo per la località scelta. & \makecell{UC0-mt}\\
\hline

ROpFu01.1-mt & \makecell{Opzionale \\ Funzionale} & Il mittente deve poter inserire il nome della località desiderata. & \makecell{UC1-mt}\\
\hline

ROpFu01.2-mt & \makecell{Opzionale \\ Funzionale} & Le previsioni meteorologiche vengono fornite tramite la libreria weather.js. & \makecell{Interno}\\
\hline

ROpFu01.3-mt & \makecell{Opzionale \\ Funzionale} & Il mittente e il ricevente devono poter visualizzare il meteo per la località selezionata. & \makecell{UC2-mt}\\
\hline

ROpFu01.4-mt & \makecell{Opzionale \\ Funzionale} & L'utente deve poter visualizzare il menu di configurazione. & \makecell{UC3-mt}\\
\hline

ROpFu01.5-mt & \makecell{Opzionale \\ Funzionale} & L'utente deve poter inviare la bolla meteo. & \makecell{UC4-mt}\\
\hline

ROpFu01.6-mt & \makecell{Opzionale \\ Funzionale} & L'utente deve poter visualizzare i risultati della ricerca effettuata utilizzando in nome della località di cui si vuole conoscere il meteo. & \makecell{UC5-mt}\\
\hline

ROpFu01.7-mt & \makecell{Opzionale \\ Funzionale} & L'utente deve poter visualizzare un opportuno messaggio di errore nel caso in cui la ricerca non abbia dato esito. & \makecell{UC7-mt}\\
\hline

ROpFu01.8-mt & \makecell{Opzionale \\ Funzionale} & L'utente deve poter selezionare il risultato desiderato tra i risultati della ricerca. & \makecell{UC6-mt}\\
\hline

ROpFu01.9-mt & \makecell{Opzionale \\ Funzionale} & L'utente non deve poter inviare la bolla prima di aver selezionato un risultato della ricerca effettuata utilizzando il nome della località di cui si vuole conoscere il meteo. & \makecell{UC4-mt}\\
\hline

RObFu01-sd & \makecell{Obbligatorio \\ Funzionale} & Il mittente può definire le opzioni tra cui i riceventi possono scegliere. & \makecell{UC1-sd}\\
\hline

RObFu02-sd & \makecell{Obbligatorio \\ Funzionale} & Il ricevente deve poter votare nel sondaggio. & \makecell{UC2-sd}\\
\hline

RObFu03-sd & \makecell{Obbligatorio \\ Funzionale} & L'utente deve poter visualizzare il menu di configurazione per la bolla sondaggio. & \makecell{UC4-sd}\\
\hline

RObFu04-sd & \makecell{Obbligatorio \\ Funzionale} & Il mittente e il ricevente devono poter visualizzare i risultati del sondaggio. & \makecell{UC3-sd}\\
\hline

RObFu05-sd & \makecell{Obbligatorio \\ Funzionale} & L'utente deve poter inviare la bolla sondaggio. & \makecell{UC5-sd}\\
\hline

RObFu06-sd & \makecell{Obbligatorio \\ Funzionale} & L'utente deve poter inserire un titolo per il sondaggio. & \makecell{UC6-sd}\\
\hline

RObFu07-sd & \makecell{Obbligatorio \\ Funzionale} & L'utente non deve poter inviare la bolla nel caso in cui non siano stati inseriti il titolo e almeno due opzioni. & \makecell{UC5-sd}\\
\hline

ROpFu08-sd & \makecell{Opzionale \\ Funzionale} & L'utente deve ricevere una notifica quando tutti i partecipanti hanno espresso il voto. & \makecell{Interno}\\
\hline

ROpFu01-tr & \makecell{Opzionale \\ Funzionale} & La bolla deve tradurre il testo inserito nelle lingue selezionate. & \makecell{UC0-tr}\\
\hline

ROpFu01.1-tr & \makecell{Opzionale \\ Funzionale} & Il mittente deve poter configurare la lingua in cui sarà scritto il messaggio e quella in cui dovrà essere tradotto. & \makecell{UC1-tr}\\
\hline

ROpFu01.2-tr & \makecell{Opzionale \\ Funzionale} & Il mittente deve poter inserire il testo da tradurre. & \makecell{UC2-tr}\\
\hline

ROpFu01.3-tr & \makecell{Opzionale \\ Funzionale} & L'utente deve poter visualizzare il menu di configurazione. & \makecell{UC3-tr}\\
\hline

ROpFu01.4-tr & \makecell{Opzionale \\ Funzionale} & La traduzione deve essere fornita tramite la libreria polyglot.js. & \makecell{Interno}\\
\hline

ROpFu01.5-tr & \makecell{Opzionale \\ Funzionale} & Il mittente e il ricevente devono poter visualizzare la traduzione. & \makecell{UC4-tr}\\
\hline

ROpFu01.6-tr & \makecell{Opzionale \\ Funzionale} & L'utente deve poter inviare la bolla traduttore. & \makecell{UC5-tr}\\
\hline

ROpFu01.7-tr & \makecell{Opzionale \\ Funzionale} & L'utente non deve poter inviare la bolla nel caso in cui non siano state impostate le lingue o inserito il messaggio. & \makecell{UC5-tr}\\
\hline

\hline
\end{longtable}
\captionof{table}{Requisiti funzionali}
\end{center}
\subsection{Requisiti qualitativi}

\begin{center}
\begin{longtable}{|
*{1}{>{\centering\arraybackslash}p{2.5cm}|}
*{1}{>{\centering\arraybackslash}p{2cm}|}
*{1}{>{\centering\arraybackslash}p{5cm}|}
*{1}{>{\centering\arraybackslash}p{2.5cm}|}}
\hline \textbf{Requisito} & \textbf{Tipologia} & \textbf{Descrizione} & \textbf{Fonti}\\
\hline \endhead
\hline \endfoot

RObQu04.1 & \makecell{Obbligatorio \\ Qualitativo} & Javascript deve essere usato secondo le Airbnb style guide. & \makecell{Capitolato}\\
\hline

RObQu06 & \makecell{Obbligatorio \\ Qualitativo} & Monolith e le bolle devono essere realizzati secondo il 12 Factors app guidelines. & \makecell{Capitolato}\\
\hline

RObQu08 & \makecell{Obbligatorio \\ Qualitativo} & Monolith deve essere corredato da una versione sintetica del manuale in inglese. & \makecell{Capitolato}\\
\hline

RObQu08.1 & \makecell{Obbligatorio \\ Qualitativo} & Il manuale deve spiegare come installare Monolith in ambiente Rochet.Chat. & \makecell{Capitolato\\Interno}\\
\hline

RObQu16 & \makecell{Obbligatorio \\ Qualitativo} & La documentazione formale standard deve essere

scritta in italiano. & \makecell{Capitolato\\Verbale20170301}\\
\hline

RObQu17 & \makecell{Obbligatorio \\ Qualitativo} & La bolla presentata come Demo deve essere documentata. & \makecell{Interno}\\
\hline

\hline
\end{longtable}
\captionof{table}{Requisiti qualitativi}
\end{center}
\subsection{Requisiti dichiarativi}

\begin{center}
\begin{longtable}{|
*{1}{>{\centering\arraybackslash}p{2.5cm}|}
*{1}{>{\centering\arraybackslash}p{2cm}|}
*{1}{>{\centering\arraybackslash}p{5cm}|}
*{1}{>{\centering\arraybackslash}p{2.5cm}|}}
\hline \textbf{Requisito} & \textbf{Tipologia} & \textbf{Descrizione} & \textbf{Fonti}\\
\hline \endhead
\hline \endfoot

RObDi01 & \makecell{Obbligatorio \\ Dichiarativo} & Monolith deve essere realizzato come pacchetto RocketChat. & \makecell{Capitolato}\\
\hline

RObDi02 & \makecell{Obbligatorio \\ Dichiarativo} & Devono essere fornite alcune bolle predefinite. & \makecell{Capitolato}\\
\hline

RObDi02.1 & \makecell{Obbligatorio \\ Dichiarativo} & Una delle bolle predefinite funge da demo al fine di dimostrare l'utilizzo delle API. & \makecell{Capitolato}\\
\hline

RObDi03 & \makecell{Obbligatorio \\ Dichiarativo} & Monolith deve includere un set di API per lo sviluppo di bolle. & \makecell{Capitolato}\\
\hline

RObDi04 & \makecell{Obbligatorio \\ Dichiarativo} & Monolith deve essere realizzato usando Javascript ES6. & \makecell{Capitolato}\\
\hline

RObDi05 & \makecell{Obbligatorio \\ Dichiarativo} & Monolith deve essere realizzato usando SCSS. & \makecell{Capitolato}\\
\hline

RObDi07 & \makecell{Obbligatorio \\ Dichiarativo} & Utilizzo del framework frontend REACT per la creazione delle interfacce. & \makecell{Capitolato}\\
\hline

RObDi07.1 & \makecell{Obbligatorio \\ Dichiarativo} & I dati dei componenti React devono essere reattivi all'interno del data-system di Meteor. & \makecell{Interno}\\
\hline

RObDi09 & \makecell{Obbligatorio \\ Dichiarativo} & La demo deve essere installabile su Heroku. & \makecell{Capitolato}\\
\hline

RObDi11.2.1.2.4 & \makecell{Obbligatorio \\ Dichiarativo} & I layout offerti dai contenitori sono realizzati utilizzando il grid system di Bootstrap. & \makecell{Interno\\UC2.1.5}\\
\hline

RObDi12 & \makecell{Obbligatorio \\ Dichiarativo} & Il codice sorgente di Monolith e delle bolle deve essere disponibile su GitHub. & \makecell{Capitolato}\\
\hline

RObDi13 & \makecell{Obbligatorio \\ Dichiarativo} & Monolith deve supportare i browser in cui è eseguibile Rocket.Chat. & \makecell{Interno}\\
\hline

RObDi14 & \makecell{Obbligatorio \\ Dichiarativo} & L'interfaccia del sistema Monolith deve essere responsive. & \makecell{Interno}\\
\hline

RObDi15 & \makecell{Obbligatorio \\ Dichiarativo} & Il framework è realizzato con uso di promise per la programmazione asincrona. & \makecell{Capitolato}\\
\hline

RObDi18 & \makecell{Obbligatorio \\ Dichiarativo} & L'utilizzo di callbacks nel codice Javascript deve essere giustificato. & \makecell{Capitolato}\\
\hline

RObDi19 & \makecell{Obbligatorio \\ Dichiarativo} & La modifica del nome delle classi HTML dei componenti è realizzata attraverso l'utility classnames. & \makecell{Interno}\\
\hline

\hline
\end{longtable}
\captionof{table}{Requisiti dichiarativi}
\end{center}
\subsection{Riepilogo requisiti}

I 148 requisiti individuati si suddividono come segue:
\begin{center}
  \centering
  \begin{tabular}{|l|c|c|c|}
    \hline
      & Funzionali & Qualitativi & Dichiarativi   \\
\hline
Obbligatori &      96     &    6     & 16       \\
\hline
Desiderabili &     4     &     0     & 0    \\
\hline
Opzionali   &      26     &    0     & 0    \\
\hline
  \end{tabular}
  \captionof{table}{Riepilogo del numero di requisiti individuati.}
\end{center}

