\subsection{Requisiti funzionali}

\begin{center}
\begin{longtable}{|
*{1}{>{\centering\arraybackslash}p{2.5cm}|}
*{1}{>{\centering\arraybackslash}p{2cm}|}
*{1}{>{\centering\arraybackslash}p{5cm}|}
*{1}{>{\centering\arraybackslash}p{2.5cm}|}}
\hline \textbf{Requisito} & \textbf{Tipologia} & \textbf{Descrizione} & \textbf{Fonti}\\
\hline \endhead
\hline \endfoot

RObFu11 & \makecell{Obbligatorio \\ Funzionale} & Lo sviluppatore deve avere a disposizione gli strumenti per creare una bolla funzionante. & \makecell{UC0}\\
\hline

RObFu11.1 & \makecell{Obbligatorio \\ Funzionale} & Lo sviluppatore deve poter creare le proprie bolle a partire dalla bolla vuota. & \makecell{UC1}\\
\hline

RObFu11.2 & \makecell{Obbligatorio \\ Funzionale} & Lo sviluppatore deve poter utilizzare le funzionalità offerte dal sistema per descrivere l'aspetto visuale delle bolle. & \makecell{UC2.1}\\
\hline

RObFu11.2.1 & \makecell{Obbligatorio \\ Funzionale} & Lo sviluppatore deve poter inserire un elemento grafico. & \makecell{UC2.1.1}\\
\hline

RObFu11.2.1.1 & \makecell{Obbligatorio \\ Funzionale} & Lo sviluppatore deve poter inserire un elemento testo. & \makecell{UC2.1.1}\\
\hline

RObFu11.2.1.2 & \makecell{Obbligatorio \\ Funzionale} & Lo sviluppatore deve poter inserire un elemento immagine. & \makecell{UC2.1.1}\\
\hline

RObFu11.2.1.3 & \makecell{Obbligatorio \\ Funzionale} & Lo sviluppatore deve poter inserire un elemento campo di inserimento testo. & \makecell{UC2.1.1}\\
\hline

RObFu11.2.1.4 & \makecell{Obbligatorio \\ Funzionale} & Lo sviluppatore deve poter inserire un elemento pulsante. & \makecell{UC2.1.1}\\
\hline

RObFu11.2.1.5 & \makecell{Obbligatorio \\ Funzionale} & Lo sviluppatore deve poter inserire un elemento checkbox. & \makecell{UC2.1.1}\\
\hline

RObFu11.2.1.6 & \makecell{Obbligatorio \\ Funzionale} & Lo sviluppatore deve poter inserire un elemento radiobutton. & \makecell{UC2.1.1}\\
\hline

RObFu11.2.2 & \makecell{Obbligatorio \\ Funzionale} & Lo sviluppatore deve poter inserire un contenitore. & \makecell{UC2.1.1}\\
\hline

RObFu11.2.3 & \makecell{Obbligatorio \\ Funzionale} & Lo sviluppatore deve poter modificare le proprietà di un elemento grafico. & \makecell{UC2.1.2}\\
\hline

RObFu11.2.3.1 & \makecell{Obbligatorio \\ Funzionale} & Lo sviluppatore deve poter impostare le dimensioni di tutti gli elementi grafici. & \makecell{UC2.1.2.1}\\
\hline

RObFu11.2.3.1.1 & \makecell{Obbligatorio \\ Funzionale} & In orizzontale le dimensioni dell'elemento vanno impostate in percentuale rispetto alle dimensioni del contenitore padre. & \makecell{UC2.1.2.1}\\
\hline

RObFu11.2.3.1.2 & \makecell{Obbligatorio \\ Funzionale} & In verticale le dimensioni dell'elemento vanno fornite in modo non dipendente dalle dimensioni del contenitore padre. & \makecell{UC2.1.2.1}\\
\hline

RObFu11.2.3.2 & \makecell{Obbligatorio \\ Funzionale} & Lo sviluppatore deve poter modificare il contenuto di un elemento testo. & \makecell{UC2.1.2.2}\\
\hline

RObFu11.2.3.3 & \makecell{Obbligatorio \\ Funzionale} & Lo sviluppatore deve poter impostare il percorso dell'immagine da visualizzare in un elemento immagine. & \makecell{UC2.1.2.3}\\
\hline

RObFu11.2.3.4 & \makecell{Obbligatorio \\ Funzionale} & Lo sviluppatore deve poter modificare il testo mostrato nel pulsante. & \makecell{UC2.1.2.4}\\
\hline

RObFu11.2.3.5 & \makecell{Obbligatorio \\ Funzionale} & Lo sviluppatore deve poter impostare l'azione da associare al pulsante. & \makecell{UC2.1.2.5}\\
\hline

RObFu11.2.3.6 & \makecell{Obbligatorio \\ Funzionale} & Lo sviluppatore deve poter impostare il testo della checkbox. & \makecell{UC2.1.2.6}\\
\hline

RObFu11.2.3.7 & \makecell{Obbligatorio \\ Funzionale} & Lo sviluppatore deve poter impostare le opzioni di scelta di un radiobutton. & \makecell{UC2.1.2.7}\\
\hline

RObFu11.2.4 & \makecell{Obbligatorio \\ Funzionale} & Lo sviluppatore deve poter eliminare un elemento grafico. & \makecell{UC2.1.3}\\
\hline

RObFu11.2.5 & \makecell{Obbligatorio \\ Funzionale} & Lo sviluppatore deve poter modificare un contenitore. & \makecell{UC2.1.4}\\
\hline

RObFu11.2.5.1 & \makecell{Obbligatorio \\ Funzionale} & Lo sviluppatore deve poter impostare l'orientamento dei sottoelementi del contenitore. & \makecell{UC2.1.4.1}\\
\hline

RObFu11.2.5.1.1 & \makecell{Obbligatorio \\ Funzionale} & Allineamento orizzontale. Lo sviluppatore deve poter elencare gli elementi da inserire e questi saranno distribuiti uniformemente nello spazio disponibile. Nel caso in cui sia specificata la dimensione di qualche sottoelemento la disposizione viene modificata di conseguenza. 



Il sistema fornisce un'interfaccia al grid system di bootstrap. & \makecell{UC2.1.4.1}\\
\hline

RObFu11.2.5.1.2 & \makecell{Obbligatorio \\ Funzionale} & Allineamento verticale. Gli elementi vengono impilati verticalmente secondo l'ordine di inserimento.

Il sistema fornisce un'interfaccia al grid system di bootstrap. & \makecell{UC2.1.4.1}\\
\hline

RObFu11.2.5.2 & \makecell{Obbligatorio \\ Funzionale} & Lo sviluppatore deve poter impostare le dimensioni del contenitore. & \makecell{UC2.1.4.2}\\
\hline

RObFu11.2.5.2.1 & \makecell{Obbligatorio \\ Funzionale} & In orizzontale le dimensioni dei contenitori vanno espresse in valori percentuali sulla dimensione del contenitore padre. & \makecell{UC2.1.4.2}\\
\hline

RObFu11.2.5.2.2 & \makecell{Obbligatorio \\ Funzionale} & In verticale le dimensioni dei contenitori vengono calcolate solo automaticamente. & \makecell{UC2.1.4.2}\\
\hline

RObFu11.2.6 & \makecell{Obbligatorio \\ Funzionale} & Lo sviluppatore deve poter eliminare un contenitore. L'eliminazione di un contenitore porta all'eliminazione di tutti i suoi elementi o contenitori figli. & \makecell{UC2.1.5}\\
\hline

RObFu11.2.6.1 & \makecell{Obbligatorio \\ Funzionale} & Non è concesso allo sviluppatore di eliminare il contenitore principale della bolla. & \makecell{UC2.1.5\\UC2.1.7}\\
\hline

RObFu11.2.7 & \makecell{Obbligatorio \\ Funzionale} & Lo sviluppatore deve poter selezionare un'elemento grafico o un contenitore. & \makecell{UC2.1.6}\\
\hline

RObFu11.2.8 & \makecell{Obbligatorio \\ Funzionale} & Lo sviluppatore deve poter navigare l'albero di elementi e contenitori della bolla. & \makecell{UC2.1.6}\\
\hline

RObFu11.3 & \makecell{Obbligatorio \\ Funzionale} & Deve essere possibile distinguere il mittente di una bolla da tutti gli altri utenti. & \makecell{UC2.2}\\
\hline

RObFu11.4 & \makecell{Obbligatorio \\ Funzionale} & Lo sviluppatore deve poter impostare azioni a tempo. & \makecell{UC2.3}\\
\hline

RObFu11.4.1 & \makecell{Obbligatorio \\ Funzionale} & Lo sviluppatore deve poter impostare azioni da compiere a intervalli specificati. & \makecell{UC2.3.1\\UC2.3.1.1}\\
\hline

RObFu11.4.2 & \makecell{Obbligatorio \\ Funzionale} & Lo sviluppatore deve poter impostare azioni da eseguire dopo un intervallo specificato. & \makecell{UC2.3.1\\UC2.3.1.2}\\
\hline

RObFu11.4.3 & \makecell{Obbligatorio \\ Funzionale} & Lo sviluppatore deve poter impostare un'azione da associare al timer. & \makecell{UC2.3.2}\\
\hline

RObFu11.5 & \makecell{Obbligatorio \\ Funzionale} & Lo sviluppatore deve poter gestire la persistenza dei dati della bolla. & \makecell{UC2.4}\\
\hline

RObFu11.5.1 & \makecell{Obbligatorio \\ Funzionale} & Lo sviluppatore deve poter registrare dati globali accessibili da ogni istanza della bolla. & \makecell{UC2.4.1}\\
\hline

RObFu11.5.2 & \makecell{Obbligatorio \\ Funzionale} & Lo sviluppatore deve poter registrare dati propri della singola istanza di bolla. & \makecell{UC2.4.2}\\
\hline

RObFu14 & \makecell{Obbligatorio \\ Funzionale} & Viene fornito un eseguibile che a partire dal codice Monolith che descrive l'interfaccia produca il file HTML. & \makecell{Interno}\\
\hline

RObFu19 & \makecell{Obbligatorio \\ Funzionale} & Lo sviluppatore deve poter impostare diversi livelli d'accesso ai componenti delle bolle. & \makecell{UC2.2}\\
\hline

RObFu01-cv & \makecell{Obbligatorio \\ Funzionale} & L'utente può convertire importi da una valuta all'altra. & \makecell{UC0-cv}\\
\hline

RObFu01.1-cv & \makecell{Obbligatorio \\ Funzionale} & L'utente può scegliere le valute tra cui effettuare la conversione. & \makecell{UC1-cv}\\
\hline

RObFu01.2-cv & \makecell{Obbligatorio \\ Funzionale} & L'utente può inserire l'importo da convertire. & \makecell{UC2-cv}\\
\hline

RObFu01.3-cv & \makecell{Obbligatorio \\ Funzionale} & I tassi di conversione vengono forniti da una fonte esterna. & \makecell{Interno\\UC3-cv}\\
\hline

RObFu01.4-cv & \makecell{Obbligatorio \\ Funzionale} & Il mittente e il ricevente devono poter visualizzare gli importi convertiti. & \makecell{UC3-cv}\\
\hline

ROpFu02-cv & \makecell{Opzionale \\ Funzionale} & L'utente può convertire importi da valori di pacchetti azionari. & \makecell{Interno}\\
\hline

RObFu01-dd & \makecell{Obbligatorio \\ Funzionale} & La bolla estrae un numero casualmente dal range impostato. & \makecell{UC0-dd}\\
\hline

RObFu01.1-dd & \makecell{Obbligatorio \\ Funzionale} & L'utente deve poter impostare il range da cui estrarre il numero casuale. & \makecell{UC1-dd}\\
\hline

RObFu01.2-dd & \makecell{Obbligatorio \\ Funzionale} & Il mittente e il ricevente devono poter visualizzare il numero casuale generato. & \makecell{UC2-dd}\\
\hline

ROpFu01.2.1-dd & \makecell{Opzionale \\ Funzionale} & Il mittente e il ricevente devono poter visualizzare il numero casuale sotto forma di immagine (per esempio le facce di uno o più dadi). & \makecell{Interno\\UC2-dd}\\
\hline

RObFu01.3-dd & \makecell{Obbligatorio \\ Funzionale} & Il sistema genera lato server un numero casuale nel range specificato utilizzando la libreria Math inclusa in Javascript. & \makecell{Interno\\UC2-dd}\\
\hline

RObFu01-ls & \makecell{Obbligatorio \\ Funzionale} & Il mittente deve poter definire una lista da inviare. & \makecell{UC1-ls}\\
\hline

RObFu01.1-ls & \makecell{Obbligatorio \\ Funzionale} & Il mittente deve poter inserire manualmente un nuovo elemento. & \makecell{UC1.1-ls}\\
\hline

RObFu01.2-ls & \makecell{Obbligatorio \\ Funzionale} & Il mittente deve poter inserire un elemento prelevandolo da una delle liste predefinite. & \makecell{UC1.2-ls}\\
\hline

RObFu02-ls & \makecell{Obbligatorio \\ Funzionale} & Il mittente deve poter definire una lista predefinita. & \makecell{UC2-ls}\\
\hline

RObFu03-ls & \makecell{Obbligatorio \\ Funzionale} & L'amministratore deve poter definire una lista predefinita. & \makecell{Interno\\UC2-ls}\\
\hline

RObFu04-ls & \makecell{Obbligatorio \\ Funzionale} & Il mittente e il ricevente devono poter spuntare una voce dalla lista inviatagli. & \makecell{UC3-ls}\\
\hline

RObFu05-ls & \makecell{Obbligatorio \\ Funzionale} & Il mittente deve visualizzare le spunte effettuate dai riceventi. & \makecell{Interno}\\
\hline

RObFu01-mt & \makecell{Obbligatorio \\ Funzionale} & La bolla deve restituire le previsioni meteo per la località scelta. & \makecell{UC0-mt}\\
\hline

RObFu01.1-mt & \makecell{Obbligatorio \\ Funzionale} & Il mittente deve poter selezionare la località desiderata. & \makecell{UC1-mt}\\
\hline

RObFu01.2-mt & \makecell{Obbligatorio \\ Funzionale} & Le previsioni meteorologiche vengono fornite da una fonte esterna. & \makecell{Interno\\UC2-mt}\\
\hline

RObFu01.3-mt & \makecell{Obbligatorio \\ Funzionale} & Il mittente e il ricevente devono poter visualizzare il meteo per la località selezionata. & \makecell{UC2-mt}\\
\hline

RObFu01-sd & \makecell{Obbligatorio \\ Funzionale} & Il mittente può definire le opzioni tra cui i riceventi possono scegliere. & \makecell{UC1-sd}\\
\hline

RObFu02-sd & \makecell{Obbligatorio \\ Funzionale} & Il mittente deve poter terminare il sondaggio a sua discrezione. & \makecell{UC2-sd}\\
\hline

RObFu03-sd & \makecell{Obbligatorio \\ Funzionale} & Il ricevente deve poter votare nel sondaggio. & \makecell{UC3-sd}\\
\hline

RObFu04-sd & \makecell{Obbligatorio \\ Funzionale} & Il mittente deve poter votare nel proprio sondaggio. & \makecell{UC3-sd}\\
\hline

RObFu05-sd & \makecell{Obbligatorio \\ Funzionale} & Il mittente e il ricevente devono poter visualizzare i risultati del sondaggio. & \makecell{UC4-sd}\\
\hline

RObFu01-tr & \makecell{Obbligatorio \\ Funzionale} & La bolla deve tradurre il testo inserito nelle lingue selezionate. & \makecell{UC0-tr}\\
\hline

RObFu01.1-tr & \makecell{Obbligatorio \\ Funzionale} & Il mittente deve poter configurare la lingua in cui sarà scritto il messaggio. & \makecell{UC1-tr}\\
\hline

RObFu01.2-tr & \makecell{Obbligatorio \\ Funzionale} & Il mittente deve poter inserire il testo da tradurre. & \makecell{UC2-tr}\\
\hline

RObFu01.3-tr & \makecell{Obbligatorio \\ Funzionale} & Il mittente deve poter configurare la lingua in cui il messaggio sarà tradotto. & \makecell{UC3-tr}\\
\hline

RObFu01.4-tr & \makecell{Obbligatorio \\ Funzionale} & La traduzione deve essere fornita da una fonte esterna. & \makecell{Interno\\UC4-tr}\\
\hline

RObFu01.5-tr & \makecell{Obbligatorio \\ Funzionale} & Il mittente e il ricevente devono poter visualizzare la traduzione. & \makecell{UC4-tr}\\
\hline

\hline
\end{longtable}
\end{center}
\subsection{Requisiti qualitativi}

\begin{center}
\begin{longtable}{|
*{1}{>{\centering\arraybackslash}p{2.5cm}|}
*{1}{>{\centering\arraybackslash}p{2cm}|}
*{1}{>{\centering\arraybackslash}p{5cm}|}
*{1}{>{\centering\arraybackslash}p{2.5cm}|}}
\hline \textbf{Requisito} & \textbf{Tipologia} & \textbf{Descrizione} & \textbf{Fonti}\\
\hline \endhead
\hline \endfoot

RObQu04.1 & \makecell{Obbligatorio \\ Qualitativo} & Javascript deve essere usato secondo le Airbnb style guide. & \makecell{Capitolato}\\
\hline

RObQu06 & \makecell{Obbligatorio \\ Qualitativo} & Monolith e le bolle devono essere realizzati secondo il 12 Factors app guidelines. & \makecell{Capitolato}\\
\hline

RObQu08 & \makecell{Obbligatorio \\ Qualitativo} & Monolith deve essere corredato da un manuale in inglese. & \makecell{Capitolato}\\
\hline

RObQu08.1 & \makecell{Obbligatorio \\ Qualitativo} & Il manuale deve spiegare come installare Monolith in ambiente Rochet.Chat. & \makecell{Capitolato}\\
\hline

RObQu15 & \makecell{Obbligatorio \\ Qualitativo} & Il framework è realizzato con uso di promise per la programmazione asincrona. & \makecell{Capitolato}\\
\hline

RObQu16 & \makecell{Obbligatorio \\ Qualitativo} & La documentazione formale standard deve essere

scritta in italiano. & \makecell{Capitolato\\Verbale2017-03-01}\\
\hline

RObQu17 & \makecell{Obbligatorio \\ Qualitativo} & La bolla presentata come Demo deve essere documentata. & \makecell{Interno}\\
\hline

\hline
\end{longtable}
\end{center}
\subsection{Requisiti dichiarativi}

\begin{center}
\begin{longtable}{|
*{1}{>{\centering\arraybackslash}p{2.5cm}|}
*{1}{>{\centering\arraybackslash}p{2cm}|}
*{1}{>{\centering\arraybackslash}p{5cm}|}
*{1}{>{\centering\arraybackslash}p{2.5cm}|}}
\hline \textbf{Requisito} & \textbf{Tipologia} & \textbf{Descrizione} & \textbf{Fonti}\\
\hline \endhead
\hline \endfoot

RObDi01 & \makecell{Obbligatorio \\ Dichiarativo} & Monolith deve essere realizzato come pacchetto RocketChat. & \makecell{Capitolato}\\
\hline

RObDi02 & \makecell{Obbligatorio \\ Dichiarativo} & Monolith deve includere alcune bolle predefinite. & \makecell{Capitolato}\\
\hline

RObDi02.1 & \makecell{Obbligatorio \\ Dichiarativo} & Una delle bolle predefinite funge da demo al fine di dimostrare l'utilizzo delle API. & \makecell{Capitolato}\\
\hline

RObDi03 & \makecell{Obbligatorio \\ Dichiarativo} & Monolith deve includere un set di API per lo sviluppo di bolle. & \makecell{Capitolato}\\
\hline

RObDi04 & \makecell{Obbligatorio \\ Dichiarativo} & Monolith deve essere realizzato usando Javascript ES6. & \makecell{Capitolato}\\
\hline

RObDi05 & \makecell{Obbligatorio \\ Dichiarativo} & Monolith deve essere realizzato usando SCSS. & \makecell{Capitolato}\\
\hline

RObDi07 & \makecell{Obbligatorio \\ Dichiarativo} & Utilizzo del framework frontend REACT per la creazione delle interfacce. & \makecell{Capitolato\\Interno\\Verbale}\\
\hline

RObDi09 & \makecell{Obbligatorio \\ Dichiarativo} & La demo deve essere installabile su Heroku. & \makecell{Capitolato}\\
\hline

RObDi12 & \makecell{Obbligatorio \\ Dichiarativo} & Il codice sorgente di Monolith e delle bolle deve essere disponibile su GitHub. & \makecell{Capitolato}\\
\hline

RObDi13 & \makecell{Obbligatorio \\ Dichiarativo} & Monolith deve supportare i browser in cui è eseguibile Rocket.Chat. & \makecell{Interno}\\
\hline

RObDi18 & \makecell{Obbligatorio \\ Dichiarativo} & L'utilizzo di callbacks nel codice Javascript deve essere giustificato. & \makecell{Capitolato}\\
\hline

\hline
\end{longtable}
\end{center}
\subsection{Riepilogo requisiti}

I 94 requisiti individuati si suddividono come segue:
\begin{center}
  \centering
  \begin{tabular}{|l|c|c|c|}
    \hline
		& Funzionali & Qualitativi & Dichiarativi   \\
\hline
Obbligatori &      74     &    7     & 11       \\
\hline
Desiderabili &     0     &	  0     & 0    \\
\hline
Opzionali   &      2     &    0     & 0    \\
\hline
  \end{tabular}
  \captionof{table}{Riepilogo del numero di requisiti individuati.}
\end{center}

